\begin{frame}{Enteros cuadráticos}
  \begin{definition}
    Decimos que $K$ es una extensión cuadrática de $\Q$ si es de la forma $K = \Q(\sqrt{d})$ donde $d$ es un entero libre de cuadrados.
  \end{definition}
\end{frame}

\begin{frame}{Enteros cuadráticos}
  \begin{theorem}
    Sea $K = \Q(\sqrt{d})$ una extensión cuadrática, entonces su anillo de enteros es
    \[
      \cO_K \coloneqq \cO_d = \Z[\omega] = \{ a + b\omega : a,b \in \Z \} 
    \]
    donde
    \[
      \omega = \begin{cases}
        \sqrt{d} &\text{si} \ d = 4k+2 \ \text{o} \ d = 4k+3, \\
        (1+\sqrt{d})/2 &\text{si} \ d = 4k+1.
      \end{cases}
    \]
  \end{theorem}
\end{frame}


\begin{frame}{Primo en los enteros cuadráticos}
  \begin{theorem}
    Sea $p \in \Z$ primo, entonces $(p)$ se puede factorizar en $\cO_d$ de 3 formas distintas.
    \pause\bigskip
    \begin{itemize}[<+->]
      \item \emph{p se escinde}:
        \[
          (p) = \cP\cP' \qquad \cP \neq \cP'.
        \]

      \item \emph{p se es inerte}: $(p)$ es un ideal primo.
      \bigskip

      \item \emph{p se ramifica}:
      \[
        (p) = \cP^2.
      \]
    \end{itemize}
  \end{theorem}
\end{frame}



\begin{frame}{Primo en los enteros cuadráticos}
  \begin{theorem}
    Sea $p > 2$ primo, entonces $(p)$ se puede factorizar en $\cO_d$ como.
    \pause\bigskip
    \begin{itemize}[<+->]
      \item $(p) = \cP\cP'$ si y solo si \emph{$p \mid c^2 - d$ para alguna $c$} y en este caso
      \[
        \cP = (p, \sqrt{d}+c) \Eqand \cP = (p, \sqrt{d} - c)
      \]

      \item $(p)$ es un ideal primo si y solo si \emph{no existe $c$ tal que $p \mid c^2 - d$}
      \bigskip

      \item $(p) = \cP^2$ si y solo si $p \mid d$, en este caso
      \[
        \cP = (p, \sqrt{d}).
      \]
    \end{itemize}
  \end{theorem}
\end{frame}


\begin{frame}{Primo en los enteros cuadráticos}
  \begin{theorem}
    Para el caso de $p = 2$ tenemos que
    \pause\bigskip
    \begin{itemize}[<+->]
      \item Si $d = 4k+2$ entonces $(2) = \cP^2$ donde
      \[
        \cP = (2, \sqrt{d})
      \] 

      \item Si $d = 4k+3$ entonces $(2) = \cP^2$ donde
      \[
        \cP = (2, 1+\sqrt{d})
      \]
      
      \item Si $d = 8k+1$ entonces $(2) = \cP \cP'$ donde
      \[
        \cP = (2, \sqrt{d}) \Eqand \cP = (2, -\sqrt{d})
      \] 

      \item Si $d = 8k+5$ entonces $(2)$ es un ideal primo.
    \end{itemize}
  \end{theorem}
\end{frame}



\begin{frame}{Ejemplo}
  En $\cO_{-14}$ tenemos que
  \begin{align*}
    (2) &= \cP_2^2 & \cP_2 &= (2, \sqrt{-14}) \\
    (3) &= \cP_3 \cP_3'  & \cP_3 &= (2, \sqrt{-14}+1) \\
                      &  & \cP_3' &= (2, \sqrt{-14}-1) \\
  \end{align*}
  De este modo
  \[
    (12) = (2)^2 (3) = \cP^4 \cP_3 \cP_3'.
  \]
\end{frame}