\begin{frame}{Integridad}
  \begin{definition}
    Sea $\alpha \in \C$, decimos que es íntegro si existe un polinomio mónico $f \in \Z[X]$ tal que $f(\alpha) = 0$. En otras palabras, si satisface una ecuación de la forma
  \[
    \alpha^n + a_{n-1}+\cdots + a_1\alpha + a_0 = 0,\qquad a_i \in \Z.
  \]
  \end{definition}

  \pause\bigskip Sin embargo, no es tan sencillo como agregar un elemento íntegro a $\Z$.
\end{frame}


\begin{frame}{Extensiones finitas}
  \begin{definition}
    Llamaremos a $K$ una extensión finita de $\Q$ si existe algún $\theta \in \C$ integro tal que
  \[
    K = \Q(\theta) = \{f(\theta) : f \in\Q[X]\}.
  \]
  \end{definition}
\end{frame}



\begin{frame}{Anillo de enteros}
  \begin{definition}
    Sea $K$ una extensión finita de $\Q$, definiremos el \emph{anillo de enteros de $K$} como
    \[
      \cO_K = \{ \alpha \in K : \alpha \ \text{es íntegro} \}.
    \]
  \end{definition}
\end{frame}


\begin{frame}{Ejemplos}
  \begin{itemize}[<+->]
    \item Si $K = \Q(i)$ entonces $\cO_K = \Z[i]$.
    \bigskip
    
    \item Si $K = \Q(\sqrt{-5})$ entonces $\cO_K = \Z[\sqrt{-5}]$.
    \bigskip
    
    \item $\Z[1/n]$ no es un anillo de enteros, ya que $1/n$ no es íntegro.
    \bigskip

    \item Si $K = \Q(\sqrt{-3})$ entonces $\Z[\sqrt{-3}] \subsetneq \cO_K$ de echo
    \[
      \cO_K = \Z(\omega) \qquad\text{donde}\qquad \omega = \frac{-1+i\sqrt{3}}{2}
    \]
  \end{itemize}
\end{frame}


\begin{frame}{Anillo de enteros}
  \begin{itemize}[<+->]
    \item Si $K = \Q(\theta)$ no necesariamente $\cO_K = \Z[\theta]$.
    \bigskip

    \item Sin embargo, existirán algunos $\alpha_1,\cdots,\alpha_n$ tales que
    \[
      \cO_K = \Z[\alpha_1,\ldots,\alpha_n] = \alpha_1\Z + \cdots +\alpha_n\Z.
    \]

    \item $\cO_K$ es de factorización única si y solo si $\cO_K$ es de ideales principales.
    \bigskip

    \item $\cO_K$ \emph{no necesariamente} es de factorización única.
  \end{itemize}
\end{frame}

\begin{frame}{Anillo de enteros}
  \begin{definition}
    Sea $\cP$ un ideal, decimos que es \emph{primo} si satisface que
    \[
      ab \in \cP \implies a \in \cP \eqor b \in \cP.
    \]
  \end{definition}

  \pause\bigskip
  \begin{definition}
    Sean $\cA$ y $\cB$ ideales, definimos su producto como el ideal
    \[
      \cA\cB = \Bigl\{ \sum_{i=1}^n a_i b_i :  a_i \in \cA,  b_i \in \cB \Bigr\}.
    \]
  \end{definition}
\end{frame}


\begin{frame}{Anillo de enteros}
  \begin{theorem}
    Sea $\cO_K$ un anillo de enteros. Para todo ideal $\cA$ existen únicos $\cP_1, \ldots, \cP_r$ ideales primos (salvo orden) tales que
    \[
      \cA = \cP_1^{v_1} \cdots \cP_r^{v_r}
    \]
  \end{theorem}
\end{frame}