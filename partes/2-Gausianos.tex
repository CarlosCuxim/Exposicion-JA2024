\begin{frame}
  \begin{center}\fontsize{20}{24}\selectfont
    ¿Por que deberíamos generalizar a los enteros?
  \end{center}
\end{frame}


\begin{frame}{Teorema de Fermat sobre la suma de dos cuadrados}
  Los primeros primos que pueden ser escritos como la suma de dos cuadrados son
  \begin{align*}
    2 &= 1 + 1, &
    5 &= 1 + 2^2, &
    13 &= 2^2 + 3^2, \\
    17 &= 1 + 4^2 &
    29 &= 2^2 + 5^2 &
    37 &= 1 + 6^2.
  \end{align*}

  \pause\bigskip
  Todos, excepto el 2, es de la forma $4k + 1$.
\end{frame}

\begin{frame}{Teorema de Fermat sobre la suma de dos cuadrados}
  \begin{theorem}[Fermat]
    Para todos los primos $p \neq 2$ se cumple que 
    \[
      p = a^2 + b^2 \iff p = 4k + 1.
    \]
  \end{theorem}

  \pause\bigskip
  Notemos que si $p = a^2 + b^2$, entonces se debe cumplir que
  \[
    p = (a + ib)(a - ib)
  \]
  donde $i = \sqrt{-1}$.
\end{frame}

\begin{frame}{Enteros Gausianos}
  \begin{definition}
    Llamamos al conjunto $\Z[i] = \{a + bi : a,b \in \Z\}$ los \emph{enteros gausianos}.
  \end{definition}

  \pause\bigskip

  \begin{theorem}
    Los enteros gausianos son un dominio euclideano considerando la siguiente función 
    \[
      N(a + bi) = a^2 + b^2.
    \]
  \end{theorem}
\end{frame}



\begin{frame}{Enteros Gausianos}
  \begin{theorem}
    Las unidades en $\Z[i]$ son $\Z[i]^\times = \{1, i, -1, -1\}$.
  \end{theorem}

  \begin{theorem}
    Los elementos primos en $\Z[i]$ están dados (salvo multiplicación por una unidad) por
    \begin{enumerate}
      \item $ \pi = 1 + i$.
      \item $ \pi = a + bi$ donde 
      \[
        a^2 + b^2 = p, \qquad p = 4k+1 \Eqand a > \abs{b} > 0.
      \]
      \item $\pi = p$ donde $p = 4k + 3$.
    \end{enumerate}
  \end{theorem}
\end{frame}

\begin{frame}{Teorema de Fermat sobre la suma de dos cuadrados}
  Así $p$ es de la forma $p = a^2 + b^2$ \pause si y solo si $p$ se factoriza en $\Z[i]$ como
  \[
    p = (a+bi)(a-bi),
  \]
  \pause que se cumple si y solo si $p$ no es primo en $\Z[i]$, \pause que, por el teorema anterior, es si y solo si $p = 4k+1$.
\end{frame}