\begin{frame}{Generalizando los enteros}
  Antes de generalizar tenemos que responder
  \begin{itemize}[<+->]
    \item ¿Dónde vamos a generalizarlo?\bigskip
    
    \item ¿Por qué queremos generalizarlo? \bigskip
     
    \item ¿Cómo vamos a generalizarlo? \bigskip
    
    \item ¿Qué propiedades vamos a preservar? \bigskip
  \end{itemize}
\end{frame}


\begin{frame}{¿Cómo?}
  Lo más sencillo es ``agregar cosas'' a los enteros. Es decir, tomar
  \[
    \Z[x_1, \ldots, x_n] = \{ f(x_1, \ldots, x_n) : f \ \text{es un polinomio de coeficientes enteros} \}
  \]
  donde $x_1, \ldots, x_n$ son elementos de ``algún conjunto''.

  \pause\bigskip

  En los enteros gausianos estamos agregando $i = \sqrt{-1}$.
\end{frame}


\begin{frame}{Enteros de Eisenstein}
  Si $\omega = (-1+i\sqrt{3})/2$ entonces $\Z[\omega]$ son los \emph{enteros de Eisenstein}.

  \pause\bigskip
  \begin{itemize}[<+->]
    \item $\Z[\omega] = \{ a+b\omega : a,b \in \Z \}$.
    \bigskip
    
    \item Las unidades son $\Z[\omega]^\times = \{ 1, \omega, \omega^2, -1, -\omega, -\omega^2 \}$.
    \bigskip

    \item Es un dominio euclideano usando $N(a+b\omega) = a^2 -ab + b^2$.
  \end{itemize}
\end{frame}



\begin{frame}{Enteros de Eisenstein}
  Los primos de $\Z[\omega]$ (salvo multiplicación por una unidades) son
  \begin{itemize}[<+->]
    \item Los primos enteros de la forma $3k+2$.
    \bigskip
    
    \item Los factores $\pi, \pi' \in \Z[\omega]$ de un primo enteros $p$ tal que
    \[
      p = \pi \pi' \Eqand p = 3k+2.
    \]

    \item El elemento $1+2\omega$.
  \end{itemize}
\end{frame}



\begin{frame}{Localización por un entero}
  Consideremos $\Z[1/n]$ donde $n \in \Z$. Es llamado la \emph{localización de $\Z$ en $n$}.

  \pause\bigskip
  \begin{itemize}[<+->]
    \item $\Z[1/n] = \{a/s \in \Q : s = n^k, k \in\Z \}$.
    \bigskip

    \item Las unidades son $\Z[1/n] = \{a/s \in \Q : a \mid n^k, k \in\Z \}$.
    \bigskip

    \item Es un dominio euclideano.
    \bigskip
    
    \item Los primos son los $p \in \Z$ tales que $p \nmid n$.
  \end{itemize}
\end{frame}



\begin{frame}{Polinomios enteros}
  Si agregamos una variable $X$ a $\Z$ obtenemos a los polinomios enteros

  \pause\bigskip
  \begin{itemize}[<+->]
    \item Las unidades son $1$ y $-1$.
    \bigskip

    \item Es un dominio de factorización única, pero no de ideales principales.
  \end{itemize}
\end{frame}


\begin{frame}{Dominio sin factorización única}
  Consideremos $\Z[\sqrt{-5}]$ se cumple que

  \pause\bigskip
  \begin{itemize}[<+->]
    \item $\Z[\sqrt{-5}] = \{a + b\sqrt{-5} : a,b \in \Z\}$.
    \bigskip

    \item Las unidades son $\Z[\sqrt{-5}]^\times = \{-1, 1\}$.
    \bigskip

    \item No es de factorización única, ya que
      \[
        6 = (1+\sqrt{-5})(1-\sqrt{-5}) = 2 \cdot 3.
      \]
  \end{itemize}
\end{frame}


\begin{frame}{Generalizando los enteros}
  \begin{itemize}
    \item<1-> ¿Dónde vamos a generalizarlo?
    
    \uncover<2->{\emph{Sobre los complejos}}
    \bigskip
    
    \item<3-> ¿Por qué queremos generalizarlo?

    \uncover<4->{\emph{Para resolver ecuaciones diofánticas}}
    \bigskip

    \item<5-> ¿Cómo vamos a generalizarlo?
    
    \uncover<6->{\emph{Agregando elementos ``especiales'' a los enteros}}
    \bigskip

    \item<7-> ¿Qué propiedades vamos a preservar?
    
    \uncover<8->{\emph{Integridad}}
  \end{itemize}
\end{frame}
